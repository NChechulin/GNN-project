\section{Introduction}

\subsection{Relevance}
The field of research (graph neural networks) might be considered relatively new, and, therefore, there is a huge number of possible improvements to be made to existing models and approaches.
Our ultimate goal is to improve the accuracy of node and graph classification.

For example, one of the proposed changes is to modify a Laplacian in such a way that it does not break existing model and improves it.
Our initial results have shown that our approach indeed works well on Karate club dataset, where we had to classify nodes, as shown on Figure~\ref{fig:laplacian_graph}.

\begin{figure}[h]
	\centering
	\includegraphics[width=0.6\textwidth]{custom_laplacian_accuracy.jpg}
	\caption{Default Laplacian (in blue) versus our Laplacian (in red). Y-axis is the accuracy, X-axis is the number of epochs}
	\label{fig:laplacian_graph}
\end{figure}

Despite already impressive accuracies achieved by other researchers, we also want to focus on learning time, since this characteristic is crucial for production use of models.
As we will show later, they are used in a wide variety of tasks, some of which require awaiting of response.
One of the techniques proposed can speed up learning process and prediction speed, while not dropping accuracy drastically.


The relative novelty of the field allows us to find new experiments and techniques.

\subsection{Subject of research}
Let us explain the tasks we can solve using GNNs \emph{Graph Neural Networks} in more detail.
There are three of them:
\begin{itemize}
	\item \textit{Node classification} --- given a graph with several labelled nodes and several classes predict a class of an unlabelled node.
	      For example, the CORA~\cite{cora_dataset} contains information about scientific papers.
	      Each paper belongs to exactly one of seven given classes.
	      Our task is to determine the class of a new paper given its feature vector and connections with other papers (in this case the edges of the graph are citations, the edges are undirected).
	\item \textit{Graph classification} --- determine type of graph.
	      A good example of this task is presented classification of molecules~\cite{how_powerful_are_gnns} or image classification~\cite{benchmarking_gnns}, as well as in many other fields.
	\item \textit{Link prediction} --- determine if two given nodes should have an edge between them.
	      A simple example could be friends suggestions in a social network.
\end{itemize}

In our research we will only consider the first two problems.

As we established, we want to consider several changes in order to improve the accuracy.
The tweaks we propose include but are not limited to:
\begin{itemize}
	\item \textit{Altering the way we compute Laplacian} --- a characteristic matrix of a graph.
	\item \textit{Edge embeddings}
	\item \textit{Using connectivity over simplices of higher dimension}. This means that in some cases we might want to consider a group of nodes as a separate object, therefore, increasing the connectivity factor. We will only work with 3-simplices:
	      \begin{figure}[h]
		      \begin{multicols}{2}
			      \centering
			      \includegraphics[width=0.2\textwidth]{simplex_1.png}
			      \caption{A part of some graph}\label{fig:clique_merged}

			      \centering
			      \includegraphics[width=0.2\textwidth]{simplex_2.png}
			      \caption{Three nodes from the left image united in 3-simplex having properties of the initial vertices}
		      \end{multicols}
	      \end{figure}
\end{itemize}

