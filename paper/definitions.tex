\section{Definitions}

\textbf{CGN} \textit{(Convolutional Graph Neural Network)} --- A type of GNN which generalizes the convolution operation to graphs.
Often we encounter convolution while we work with grid-structured data like images, but here we use same idea (aggregate features of the neighbors) on nodes instead of pixels\cite{9046288}.

\begin{figure}[h]
    \begin{multicols}{2}
        \centering
        \includegraphics[width=0.25\textwidth]{conv.png}
        \caption{Convolution on image}
    
        \centering
        \includegraphics[width=0.25\textwidth]{CGN_conv.png}
        \caption{Convolution on graph}
    \end{multicols}
\end{figure}

\textbf{GAT} \textit{(Graph Attention Network)} --- A type of GNN which uses attention mechanism (also borrowed from `casual' neural networks) which allows us to work with inputs of variable sizes and to focus on the most important features~\cite{velickovic2018graph}. 

\textbf{Laplacian matrix} --- A matrix representation of a graph.
Usually is calculated using the following formula~\cite{wiki_laplacian}:
\begin{equation*}
    L_{i, j} = 
    \begin{cases}
        \deg(v_i) & \mbox{if}\ i = j \\
        -1 & \mbox{if}\ i \neq j\ \mbox{and}\ v_i \mbox{ is adjacent to } v_j \\
        0 & \mbox{otherwise},
\end{cases}
\end{equation*}
