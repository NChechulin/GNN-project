\section{Work schedule}

\subsection{Done}

At this point, I have understood the subject area by thoroughly reading the papers mentioned in the reference section.
Also, I started implementing my own implementation of a Graph Convolutional Network from scratch (that is, without any frameworks at all)~\cite{gcn_implementation}.

We made this decision in order to be able to dig even deeper into the topic.
Another benefit is the fact that improves the flexibility, since it is much easier to experiment with code written by yourself, rather than try to modify existing models.

\subsection{Future plans}

\begin{enumerate}
	\item \textit{Finish implementing the model}.
	      It still lacks some crucial modules to work, however, we are almost at the finish line.
	      The only part missing at the moment is the backpropagation logic.
	      This should be done by the end of February 2022.

	\item \textit{Find additional datasets}.
	      There are several datasets which are used for benchmarks~\cite{cora_dataset}~\cite{karate_club_dataset}, however, it is always great to find new ones in order to see how existing models perform and how our tweaks influence the results.
	      This is an ongoing task, so it does not have a particular deadline.

	\item \textit{Generalization of Laplacian}.
	      Extraordinary simple to implement as soon as our model is done, but requires some time finding the generalizations.

	\item \textit{Work with simplices of higher dimensions}.
	      This requires development of some kind of preprocessing mechanism, which takes graph as an input and provides the neural network with another graph.
	      Efficiency is important, so the development might take significant time.
	      I estimate the deadline to be in the middle of March 2022.

	\item \textit{Come up with new experiments}
\end{enumerate}